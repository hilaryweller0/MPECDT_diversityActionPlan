\documentclass[12pt]{article}
\usepackage[margin=2.5cm]{geometry}
\usepackage{times,url}
\begin{document}

\title{MPECDT Draft Action Plan for Equality, Diversity and Inclusion}
\maketitle

\noindent
Our ambition is to attract students with the greatest potential to make a contribution in mathematics and to have a wide reaching impact in atmospheric or oceanic science. We need students from all backgrounds and with the full range of diverse characteristics. We have had some success in achieving gender balanced cohorts and attracting minority ethnic students. This action plan lays down what we are already doing and will continue to do and what we plan to do to improve our diversity.

This action plan has been drawn up in consultation with the EPSRC action plan:
\begin{quote}
\url{https://www.epsrc.ac.uk/funding/equalitydiversity/}
\end{quote}
in line with the equality policies at Imperial College:
\begin{quote}
\url{http://www.imperial.ac.uk/equality/equality-at-imperial/policies/}
\end{quote}
and in line with the equality and diversity policies at the University of Reading:
\begin{quote}
\url{http://www.reading.ac.uk/diversity/} \\
\url{http://www.smpcs.reading.ac.uk/equality-and-diversity/}
\end{quote}

\begin{enumerate}
\item We will review our website, http://mpecdt.org/, to ensure that it appears welcoming to a diverse student population, including photographs showcasing the diversity of our student body and avoiding potentially intimidating aspirational language. There are currently a number of ways in which our website could be improved:

\begin{enumerate}
\item We currently only have photos on our landing page which shows only one photo at a time. This photo might be of people or of science. Most of the photos of people are posed. Most browsers use a wide window and our website is narrow so there would be space to have photos of people socialising and studying down the side of each page.

\item Current students have reported that some of the advice on the website is intimidating, for example some of the ``Useful Mathematics and Tools'' on the admissions page are unrealistic even for good UK mathematics graduates. This will be removed.

\end{enumerate}

The directors will appoint someone to take action to improve our website. The first task is to engage a photographer to capture the diversity of our current cohorts engaging in CDT activities. Next we need to employ a consultant to improve our website, including the new photographs. 

\item Our recruitment policy is in line with the University of Reading selection and recruitment policy which aims to recruit students with the most potential rather than students who have had the most opportunities in life. This ambition is articulated on our admissions and equality and diversity web pages:
\begin{quote}
\url{http://mpecdt.org/admissions/} \\
\url{http://mpecdt.org/equality-and-diversity/}
\end{quote}
Aspects of our admissions policy are particularly relevant to equality and diversity:
\begin{enumerate}
\item The website asks candidates to declare if they think that ``a lack of previous opportunities has adversely affected your achievements to date''. This will help us to find candidates with the most potential.
\item All applications are viewed by the admissions secretary and the admissions tutors at Reading and Imperial before any decisions are made to reject or invite for interview. This helps to ensure that the unconscious bias of a single person does not compromise an application. 
\item The applications of candidates selected for interview are reviewed by two further staff and ranked before interview. This helps to ensure consistency.
\item Candidates are interviewed by a panel of two members of staff. This panel will include a woman if the candidate is female. 
\item All candidates are asked the same questions at interview.
\item All interview candidates have been asked in advance to advise if they require any ``Reasonable adjustments'' in relation to the interview process to ensure fair treatment.
\end{enumerate}

\item We will redouble our efforts to invite a diverse range of speakers and ensure that at least 30\% of our Wednesday speakers are female or male. The directors will be appoint someone to be responsible for this and they will report annually.

\item We will ensure that we do not hold events during school holidays as this makes it difficult for staff and students with child caring responsibilities to travel. This action is based on a student comment (see appendix).

\item We will continue to have annual equality and diversity meetings between the students and the E\&D officers from both universities. These have proved to be a safe place for students to share their experiences of, for example, sexist comments from staff. The students also make useful suggestions about how to improve recruitment and improve the working environment. Some feedback from these meeting is in the appendix.

\item We are aware of some marginalising language that has, on rare occasions, been used by staff. Examples are:
\begin{enumerate}
\item Referring to the students as ``children''.
\item Referring to students as ``girls'' or ``boys''.
\item Criticising the students for choosing uni-cultural groups for group work.
\item Comments to female students that it must have been easier for them to be accepted into the CDT (from male and female members of staff).
\end{enumerate}
Students were understandably unwilling to make formal complaints as this would damage relationships between staff and students. We therefore need an easier route by which staff can be alerted to the fact that their comments can be marginalising. Our plan is to raise these issues at staff meetings without naming students. The purpose will not be to shame or reprimand staff but just to raise awareness of unintentional marginalisation. We hope that we will continue to be made aware of these incidents through the annual diversity meetings, staff-students committees and social interactions between staff and students. 

\item We will continue to advertise on national and international websites, mailing lists and through our network of contacts. We will make sure that the reach the broadest range of students by continuing to request UK mathematics departments to display our posters. 

\item We will encourage staff to attend diversity awareness training and online unconscious bias training. Both universities provide and keep records of this training. 

\item We will continue to allow flexible working for our students. The provision of laptops to all students helps this process. This is articulated on our equality and diversity web page.

\item We will encourage students to take online equality and diversity training provided by EPSRC.

\item In our renewal bid we will ask for funding to support the equality and diversity work described in this action plan.

\item In our renewal bid, we will ask for travel money to support the additional child care costs associated with academic travel for staff and students. 

\item We will set targets for the gender balance of the CDT staff. We successfully appointed a new female deputy director as a result of targetting potential applicants. Our advertisements for new positions will highlight our desire to employ women, BAME candidates and those who require flexible working arrangements. Our experience is that this can be helpful. 

\item We aim to join with more social and academic diversity events at Imperial and Reading such as talks about diversity, black history week events and gay pride. The directors will appoint someone to look out for and publicise these events within the CDT.

\item We will continue to monitor the diversity of our cohort against four protected characteristics, compare with similar student bodies at Imperial and Reading and HESA figures. These figures are reported in table \ref{tab:protectedCharStats}. Based on these figures, we can make targets. These statistics and targets will be published on our website. We feel that we have made sufficient progress improving the gender bias of our cohort in order to be able to set a target. We do not feel that we understand the statistics about the other protected characteristics well enough to set targets. For example, we suspect that the reason why we have fewer BAME students is that we have fewer international students than the other cohorts at Reading and Imperial and that this is due to the nature of our funding (we have so far not taken on any self-funded students). 

\begin{table} \centering
\begin{tabular}{l|c|c|c|c|c}
        & MPECDT & Reading & Imperial & HESA & Target\\ \hline\hline
Female  & 33\% & 36\% & 19\% & 46\% & 40\% \\
Over 30 & 3\%  & 23\% & 10\% & 22\% & -- \\
BAME    & 8\%  & 21\% & 32\% & 20\% & ? \\
Declared Disability & 5\%  & 10\% & 4\%  & 5\%  & -- \\
\end{tabular}
\caption{Diversity of the MPECDT cohort against four protected characteristic in comparison to the Reading School of Mathematical, Physical and Computational Sciences PhD student body (Reading), the entire Mathematics PhD student body at Imperial (Imperial) and to HESA figures combining taught and research postgraduates. \label{tab:protectedCharStats}}
\end{table}

\end{enumerate} 

\appendix
\section*{Appendix: Feedback from the staff-student diversity meetings}

We have received useful feedback from students at the annual staff-student diversity meetings which is summarised here. This has been helpful in writing this action plan.

\begin{itemize}
\item There is information on the website about the mathematical background that we expect our applicants to have. The students reported that some of the more advanced topics in this list are intimidating.
\item The website is disorganised and out of date and it can be difficult to find information about previous events.
\item The students liked the big open day that was held at Imperial in 2014. MPECDT took part in this open day.
\item The students reported some sexist comments from staff. In particular:
\begin{itemize}
\item Criticising the students for choosing uni-cultural groups for group work.
\item Comments to female students that it must have been easier for them to be accepted into the CDT (from male and female members of staff).
\end{itemize}
\item CDT events should not be held during school holidays as this can make it too difficult for staff with child care responsibilities to travel. 
\item There has recently been a lack of female MPECDT Wednesday speakers.
\item We need to reach out to a wider range of universities to recruit new students, not just Oxford and Cambridge.
\item Students thought that posters in other universities were a good way to reach a diverse range of students whereas relying on recommendations from tutors could be perceived as elitist. 
\item The students recommended advertising through facebook
\item The students recommended attending the LMS and IMA women in mathematics days.
\end{itemize}

\end{document}
